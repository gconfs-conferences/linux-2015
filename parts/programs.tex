\section{Programs}
\frame{
    \center \LARGE{Programs ?}
}
\begin{frame}
    \frametitle{Builtins}
    \begin{itemize}
	\item Builtins are within your shell
	\item cat, ls, cp, cp, mv, mkdir, rm
	\item Their names are logically chosen
	\begin{itemize}
	    \item cat for Con\textbf{cat}enate
	    \item cp for \textbf{C}o\textbf{p}y
	    \item and so on
	\end{itemize}
    \end{itemize}
\end{frame}

\begin{frame}
    Insert a demo here
\end{frame}

\begin{frame}
    \frametitle{Programs}
    \begin{itemize}
	\item Your system is built with programs that are required
	\item Some distributions are built with a lot of programs for the final user (Ubuntu, Mint, etc..)
	\item Some are built with the minimum required (Arch, Gentoo, Crux, etc..)
	\item But you can install more programs with a Package Manager
    \end{itemize}
\end{frame}

\begin{frame}
    \frametitle{Combining programs}
    \begin{itemize}
	\item Redirections :
	\begin{itemize}
	    \item >
	    \item >>
	\end{itemize}
	\item Pipe : `|`
	\begin{itemize}
	    \item Pipes are not limited, eg : ls -la | grep . | grep vim
	\end{itemize}
    \end{itemize}
\end{frame}

\begin{frame}
    \frametitle{Rights}
    \begin{itemize}
	\item read
	\item write
	\item execute
	\item These rights can be modified
	\begin{itemize}
	    \item man chown chgrp chmod
	\end{itemize}
    \end{itemize}
\end{frame}

\frame{
    \center \LARGE{Can I haz root ?}
}

\begin{frame}
    \frametitle{root}
    \begin{itemize}
	\item "root" is the default user on your machine
	\item sudo
	\item su -c
    \end{itemize}
\end{frame}
